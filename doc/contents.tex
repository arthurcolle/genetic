\section{Wstęp} % (fold)
  \label{sec:wstep}

  Zadaniem projektowym było zaimplementowanie oraz zbadanie skuteczności
  algorytmu ewolucyjnego dla jednoprocesorowego problemu szeregowania zadań,
  gdzie kryterium była minimalizacja sumy ważonej czasów opóźnień.

% section wstep (end)

\section{Opis problemu} % (fold)
  \label{sec:problem}

  Problem szeregowania zadań na jednym procesorze z minimalizacją sumy ważonej
  czasów opóźnień można zdefiniować w następujący sposób.
  \vspace{1em}

  Danych jest $n$ zadań, z których każde zadanie $i$ jest dostępne w chwili $t=0$
  oraz wymaga $p_i > 0$ jednostek czasu na wykonanie, posiada oczekiwany termin
  zakończenia $d_i > 0$ oraz priorytet $w_i > 0$. Zadanie $i$ jest opóźnione,
  gdy czas zakończenia zadania $C_i > d_i$, a miarą opóźnienia zadania $i$
  jest $T_i = \max(0, C_i - d_i)$. Należy znaleźć taką kolejność wykonywania
  zadań, aby zminimalizować kryterium $TWT = \sum_{i = 1}^{n} w_i \cdot T_i$.

% section problem (end)

\section{Opis algorytmu} % (fold)
  \label{sec:algorytm}

  Algorytm genetyczny należy do grupy algorytmów heurystycznych i jest oparty
  na zjawisku ewolucji biologicznej. Algorytmy te często wykorzystuje się do
  rozwiązywania problemów NP-trudnych. Zwykle robi się tak ze względu na to, że
  czas odnalezienie rozwiązania optymalnego jest zbyt duży, a wystarczająco
  dobrym może być rozwiązanie przybliżone.
  \vspace{1em}

  Działanie algorytmu można podzielić na następujące etapy.
  \begin{enumerate}
    \item Generowanie populacji początkowej;
    \item \label{start} Wybierz elitę na podstawie funkcji dopasowania;
    \item Stwórz nowe osobniki na podstawie selekcji rodziców z elity;
    \item Z określonym prawdopodobieństwem zmutuj niektóre z nowych osobników;
    \item Jeśli nowe osobniki nie są tak liczne, jak populacja początkowa,
          dołącz elitę do nowej populacji;
    \item Jeśli nie przekroczono zadanej liczby kroków, wróć do
          punktu~\ref{start}.
  \end{enumerate}
  \vspace{1em}

  Warto zwrócić uwagę, że osobnik należący do elity w algorytmie genetycznym
  może przetrwać bardzo wiele pokoleń, co w rzeczywistości nie jest możliwe.
  \vspace{1em}

  Wybór populacji początkowej został zrealizowany poprzez generowanie losowych
  permutacji zadań, a wybór rodziców uwarunkowany jest wynikiem funkcji
  przystosowania osobników.
  \vspace{1em}

  Jako algorytm mutacji użyty został losowy ruch typu invert. Zajście mutacji
  jest opisane pewnym, małym, prawdopodobieństwem i ma zachodzić niezwykle
  rzadko.
  \vspace{1em}

  Do krzyżowania osobników wykorzystany został operator PMX, który działa w
  następujący sposób.
  \begin{enumerate}
    \item Wybierz dwa różne punkty krzyżowania od $1$ do $n$, gdzie $n$ jest
          długością permutacji;
    \item Utwórz listę dwuelementowych wektorów permutacji na podstawie list
          wyciętych z permutacji rodziców;
    \item W każdym z rodziców zamień wszystkie elementy występujące na pozycji
          pierwszej w wektorze na elementy występujące na pozycji drugiej i
          odwrotnie.
  \end{enumerate}

% section algorytm (end)

\section{Implementacja} % (fold)
  \label{sec:impl}

  Algorytm zaimplementowany został w języku Erlang. Warto zwrócić uwagę, że
  język ten nie posiada takich konstrukcji jak pętle, a zmienne po przypisaniu
  do nich wartości nie mogą zostać zmodyfikowane. Komunikacja między procesami
  opiera się na wysyłaniu wiadomości, co jest to zaletą w przypadku,
  gdy język ten używany jest w aplikacjach wielowątkowych, gdyż w wielu
  przypadkach pozwala to na uniknięcie blokad.
  \vspace{1em}

  Funkcja dopasowania została zaimplementowana w następujący sposób.
  Funkcja \texttt{lists:foldl} wykonuje funkcję podaną jako pierwszy argument na
  liście podanej jako ostatni argument i zwraca wartość akumulatora, którego
  początkowy stan jest przekazany jako drugi argument. Pojedyncze zadanie jest
  reprezentowane jako krotka w formacie \texttt{\{Pj, Wj, Dj\}}. Funkcja
  dopasowania jest jedyną funkcją, która wymusza taki format zapisu danych, a
  więc zmieniając tę jedną funkcję można przystosować algorytm do rozwiązywania
  innego problemu.
  \singlespacing
  \begin{center}
  \begin{minted}[gobble=4]{erlang}
    fitness(Permutation) ->
      {_, Result} = lists:foldl(fun compute_fitness/2, {0, 0}, Permutation),
      Result.

    compute_fitness({Pj, Wj, Dj}, {Time, Acc}) ->
      {Time + Pj, Wj*max(0, Time + Pj - Dj) + Acc}.
  \end{minted}
  \end{center}
  \onehalfspacing
  \vspace{1em}

  Wybór pierwszej populacji zrealizowany został algorytmem losowym. Funkcja ta
  została zaimplementowana przy użyciu rekurencji ogonowej. Każdy osobnik w
  populacji reprezentuje określona permutacja zadań zapisana jako lista. Funkcja
  \texttt{spawn\_population} zwraca listę list, czyli listę osobników. W
  implementacji widać również wykorzystanie jednej z cech języków funkcyjnych,
  czyli dopasowywanie do wzorca, dzięki czemu nie zostało użyte żadne wyrażenie
  warunkowe.
  \singlespacing
  \begin{center}
  \begin{minted}[gobble=4]{erlang}
    spawn_population(Tasks, N) -> spawn_population(Tasks, N, []).
    spawn_population(_, 0, Acc) -> Acc;
    spawn_population(Tasks, N, Acc) ->
      Permutation = lists:keysort(2, [{X, random:uniform()} || X <- Tasks]),
      New = [X || {X,_} <- Permutation],
      spawn_population(Tasks, N - 1, [New|Acc]).
  \end{minted}
  \end{center}
  \onehalfspacing
  \vspace{1em}

  Implementacja algorytmu PMX została podzielona na trzy części. W pierwszej z
  nich wybierane są punkty krzyżowania, w drugiej generowane są wektory
  wykorzystywane przy krzyżowaniu, a w trzeciej wykonywana jest sama operacja
  krzyżowania.
  \singlespacing
  \begin{center}
  \begin{minted}[gobble=4]{erlang}
    breed(Parents = {P1, P2}, ProbabilityOfMutation) ->
      S1 = random:uniform(length(P1)),
      S2 = random:uniform(length(P2)),
      case S1 > S2 of
        true  -> breed(Parents, S2, S1, ProbabilityOfMutation);
        false -> breed(Parents, S1, S2, ProbabilityOfMutation)
      end.

    breed(Parents = {P1, P2}, S1, S2, P) ->
      V = breed_vector(Parents, S1, S2),
      C1 = lists:map(fun(X) -> lists:foldl(fun breed_swap/2, X, V) end, P1),
      C2 = lists:map(fun(X) -> lists:foldl(fun breed_swap/2, X, V) end, P2),
      {mutate(C1, P), mutate(C2, P)}.

    breed_swap({Gene, NewGene}, Gene) -> NewGene;
    breed_swap({Gene, NewGene}, NewGene) -> Gene;
    breed_swap({_, _}, Gene) -> Gene.

    breed_vector({Parent1, Parent2}, S1, S2) ->
      L1 = lists:sublist(Parent1, S1, S2 - S1),
      L2 = lists:sublist(Parent2, S1, S2 - S1),
      lists:zip(L1, L2).
  \end{minted}
  \end{center}
  \onehalfspacing
  \vspace{1em}

  Za główną pętlę algorytmu odpowiada funkcja \texttt{evolve}. To w niej
  dokonywany jest podział populacji na elitę oraz wybranie rodziców dla nowych
  pokoleń. Dodatkowym parametrem jest również znane rozwiązanie optymalne i
  jeśli zostanie ono osiągnięte wcześniej, algorytm przerywa swoje działanie.
  \singlespacing
  \begin{center}
  \begin{minted}[gobble=4]{erlang}

    evolve(Population, TimeLeft, Pmutation, Best) ->
      Sorted = sort_by_fitness(Population),
      evolve(Sorted, TimeLeft, Pmutation, hd(Sorted), Best).

    evolve(_, 0, _, BestSolution, _) -> {BestSolution, 0};
    evolve(Population, TimeLeft, Pmutation, _, Best) ->
      Length = length(Population) div 3,
      {Good, Bad} = lists:split(Length, Population),
      NewGood = reproduce(Good, Pmutation),
      Sorted = sort_by_fitness(NewGood ++ Good ++ lists:sublist(Bad, Length)),
      BestSolution = hd(Sorted),
      case fitness(BestSolution) =< Best of
        false -> evolve(Sorted, TimeLeft - 1, Pmutation, BestSolution, Best);
        true  -> {BestSolution, TimeLeft}
      end.
  \end{minted}
  \end{center}
  \onehalfspacing
  \vspace{1em}

  Tworzenie nowych osobników populacji odbywa się w funkcji \texttt{reproduce}.
  Wywołuje ona funkcję \texttt{breed} dla każdej z par rodziców.
  \singlespacing
  \begin{center}
  \begin{minted}[gobble=4]{erlang}
    reproduce(Generation, P) -> reproduce(Generation, [], P).
    reproduce([], NewGeneration, _) -> NewGeneration;
    reproduce([P1, P2|Rest], NewGeneration, P) ->
      {C1, C2} = breed({P1, P2}, P),
      reproduce(Rest, [C1, C2|NewGeneration], P);
    reproduce([Last], NewGeneration, P) ->
      reproduce([], [Last|NewGeneration], P).
  \end{minted}
  \end{center}
  \onehalfspacing
  \vspace{1em}

  Za mutacje odpowiada funkcja \texttt{mutate}. Została zaimplementowana tak,
  aby osobnik był modyfikowany tylko z pewnym prawdopodobieństwem
  $(\frac{1}{2})^\textrm{P}$ w zależności od przekazywanego parametru.
  \singlespacing
  \begin{center}
  \begin{minted}[gobble=4]{erlang}
    mutate(Permutation, P) ->
      case probability(P) of
        true  -> mutate(Permutation);
        false -> Permutation
      end.

    mutate(Permutation) ->
      S1 = random:uniform(length(Permutation)),
      S2 = random:uniform(length(Permutation)),
      case S1 > S2 of
        true  -> mutate(Permutation, S2, S1);
        false -> mutate(Permutation, S1, S2)
      end.

    mutate(Permutation, S1, S2) ->
      {Head, Tail} = lists:split(S1, Permutation),
      {Middle, End} = lists:split(S2 - S1, Tail),
      Head ++ lists:reverse(Middle) ++ End.
  \end{minted}
  \end{center}
  \onehalfspacing
  \vspace{1em}

  Algorytm zwraca najlepsze znalezione rozwiązanie oraz liczbę generacji, która
  pozostała, która jest różna od zera, jeśli rozwiązanie optymalne udało znaleźć
  się wcześniej.
% section impl (end)

\section{Testy i wyniki} % (fold)
  \label{sec:testy}

  W programie istnieje możliwość regulowania takich parametrów, jak rozmiar
  populacji, ilość kroków oraz prawdopodobieństwo mutacji. Testy zostały
  przeprowadzone dla populacji o wielkościach 50, 100 oraz 200 osobników oraz
  dla 1000, 5000 oraz 10000 kroków. Jako ostatni parametr została przekazana
  liczba 4, co oznacza, że prawdopodobieństwo mutacji wynosiło $\frac{1}{16}$.
  \vspace{1em}

  Testy zostały przeprowadzone na serwerze wyposażonym w procesor Intel Xeon
  L5520 pracujący z częstotliwością 2.27GHz oraz 512MB pamięci RAM. Program
  został uruchamiany przez wirtualną maszynę Erlanga w wersji R14B.

% section testy (end)

\section{Wnioski} % (fold)
  \label{sec:wnioski}


% section wnioski (end)

\appendix
\section{Bibliografia} % (fold)
  \label{sec:biblio}
  \begin{enumerate}
    \item \texttt{ftp://sith.ict.pwr.wroc.pl/Informatyka/PEA/ProjektPEA\_11\_12.pdf}
  \end{enumerate}

% section biblio (end)
